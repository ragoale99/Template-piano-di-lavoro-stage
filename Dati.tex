%----------------------------------------------------------------------------------------
%   USEFUL COMMANDS
%----------------------------------------------------------------------------------------

\newcommand{\dipartimento}{Dipartimento di Matematica ``Tullio Levi-Civita''}

%----------------------------------------------------------------------------------------
% 	USER DATA
%----------------------------------------------------------------------------------------

% Data di approvazione del piano da parte del tutor interno; nel formato GG Mese AAAA
% compilare inserendo al posto di GG 2 cifre per il giorno, e al posto di 
% AAAA 4 cifre per l'anno
\newcommand{\dataApprovazione}{Data}

% Dati dello Studente
\newcommand{\nomeStudente}{Alessandro}
\newcommand{\cognomeStudente}{Rago}
\newcommand{\matricolaStudente}{1187504}
\newcommand{\emailStudente}{alessandro.rago.1@studenti.unipd.it}
\newcommand{\telStudente}{+39 346 960 7400}

% Dati del Tutor Aziendale
\newcommand{\nomeTutorAziendale}{Fabio}
\newcommand{\cognomeTutorAziendale}{Pallaro}
\newcommand{\emailTutorAziendale}{f.pallaro@synclab.it}
\newcommand{\telTutorAziendale}{+39 333 136 8500}
\newcommand{\ruoloTutorAziendale}{}

% Dati dell'Azienda
\newcommand{\ragioneSocAzienda}{Sync Lab S.r.l.}
\newcommand{\indirizzoAzienda}{Galleria Spagna 28, Padova (PD)}
\newcommand{\sitoAzienda}{https://www.synclab.it/}
\newcommand{\emailAzienda}{info@synclab.it}
\newcommand{\partitaIVAAzienda}{P.IVA 12345678999}

% Dati del Tutor Interno (Docente)
\newcommand{\titoloTutorInterno}{Prof.}
\newcommand{\nomeTutorInterno}{Ombretta}
\newcommand{\cognomeTutorInterno}{Gaggi}

\newcommand{\prospettoSettimanale}{
     % Personalizzare indicando in lista, i vari task settimana per settimana
     % sostituire a XX il totale ore della settimana
    \begin{itemize}
        \item \textbf{Prima Settimana (40 ore)}
        \begin{itemize}
            \item Presentazione strumenti di lavoro per la condivisione del materiale di studio e per la gestione dell'avanzamento;
            \item Condivisione scaletta di argomenti;
            \item Ripasso concetti Web (Servlet, servizi Rest, Json ecc.);
            \item Ripasso del framework React;
        \end{itemize}
        \item \textbf{Seconda Settimana - Sottotitolo (40 ore)} 
        \begin{itemize}
            \item Ripasso linguaggio Javascript;
            \item Studio del linguaggio TypeScript;
            \item Studio Framework Angular.
        \end{itemize}
        \item \textbf{Terza Settimana - Sottotitolo (40 ore)} 
        \begin{itemize}
            \item Implementazione maschera di login e di voto on line con Angular.
        \end{itemize}
        \item \textbf{Quarta Settimana - Sottotitolo (40 ore)} 
        \begin{itemize}
            \item Studio framework VueJS.
        \end{itemize}
        \item \textbf{Quinta Settimana - Sottotitolo (40 ore)} 
        \begin{itemize}
            \item Implementazione maschera di login e di voto on line con VueJS.
        \end{itemize}
        \item \textbf{Sesta Settimana - Sottotitolo (40 ore)} 
        \begin{itemize}
            \item Implementazione maschera di login e di voto on line con React.
        \end{itemize}
        \item \textbf{Settima Settimana - Sottotitolo (40 ore)} 
        \begin{itemize}
            \item Termine implementazioni ed analisi comparative.
        \end{itemize}
        \item \textbf{Ottava Settimana - Conclusione (40 ore)} 
        \begin{itemize}
            \item Termine analisi e collaudo finale.
        \end{itemize}
    \end{itemize}
}

% Indicare il totale complessivo (deve essere compreso tra le 300 e le 320 ore)
\newcommand{\totaleOre}{320}

\newcommand{\obiettiviObbligatori}{
	 \item \underline{\textit{O01}}: Acquisizione competenze sulle tematiche sopra descritte;
	 \item \underline{\textit{O02}}: Capacità di raggiungere gli obiettivi richiesti in autonomia seguendo il cronoprogramma;
	 \item \underline{\textit{O03}}: Portare a termine le implementazioni previste con una percentuale di superamento pari all'80\%. 
}

\newcommand{\obiettiviDesiderabili}{
	 \item \underline{\textit{D01}}: Portare a termine le implementazioni previste con una percentuale di superamento pari al 100\%.
}

\newcommand{\obiettiviFacoltativi}{
	 \item \underline{\textit{F01}}: Approfondire la conoscenza di Angular studiando la gestione degli stati (NGRX).
}